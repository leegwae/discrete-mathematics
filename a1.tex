\documentclass{article}
\usepackage{amsmath}
\usepackage{amssymb}
\usepackage{amsthm}
\usepackage{amssymb}
\usepackage{mathdots}
\usepackage[pdftex]{graphicx}
\usepackage{fancyhdr}
\usepackage[margin=1in]{geometry}
\usepackage{multicol}
\usepackage{bm}
\usepackage{listings}
\PassOptionsToPackage{usenames,dvipsnames}{color}  %% Allow color names
\usepackage{pdfpages}
\usepackage{algpseudocode}
\usepackage{tikz}
\usepackage{enumitem}
\usepackage[T1]{fontenc}
\usepackage{inconsolata}
\usepackage{framed}
\usepackage{wasysym}
\usepackage[thinlines]{easytable}
\usepackage{hyperref}
\usepackage{wrapfig}
\usepackage{graphicx}
\usepackage{scrextend}
\graphicspath{ {./images/} }
\hypersetup{
    colorlinks=true,
    linkcolor=blue,
    filecolor=magenta,
    urlcolor=blue,
}
\usepackage{kotex}
\usepackage{fancyhdr}
\pagestyle{fancy}
\fancyhf{}
\usepackage{tcolorbox}


\begin{document}
\lhead{이산수학\\Assignment #1}
\rhead{SPRING 2021}
\begin{verbatim}
학번: 2018640020
이름: 이준영
01분반/02분반 여부: 01분반
\end{verbatim}
\newline

14.
\begin{addmargin}[1em]{2em}
\begin{center}
\begin{tabular}{ |c|c|c|c|c|c|c| } 
 \hline
 p & q & \sim p & p \wedge \sim q & \sim p \vee (p \wedge \sim q) & p \wedge q & (p \wedge q) \vee (\sim p \vee (p \wedge \sim q)) \\ 
  \hline
  
 T & T & F & F & F & T & T \\
  \hline
 T & F & F & T & T & F & T \\
 \hline
 F & T & T & F & T & F & T \\
 \hline
 F & F & T & F & T & F & T \\
 \hline
\end{tabular}
\end{center}

\end{addmargin}
\bigskip

15.
\begin{addmargin}[1em]{2em}
tautology이다.

\end{addmargin}
\bigskip

16.
\begin{addmargin}[1em]{2em}
tautology는 statement variable로 치환된 개별적인 statement의 진리값에 상관없이, 언제나 참인 statement form이다. 14번의 truth table로 미루어보았을 때, statement variable p와 q의 진리값으로 가능한 모든 경우에 있어 이 statement form의 진리값은 참이므로, tautology이다.

\end{addmargin}
\bigskip

17.
\begin{addmargin}[1em]{2em}
\begin{center}
\begin{tabular}{ |c|c|c|c|c|c|c|c|c| } 
 \hline
p & q & r & \sim p & \sim p \wedge q & q \wedge r & (\sim p \wedge q) \wedge (q \wedge r) & \sim q & ((\sim p \wedge q) \wedge (q \wedge r)) \wedge \sim q \\ 
  \hline
 T & T & T & F & F & T & F & F & F\\
  \hline
 T & T & F & F & F & F & F & F & F\\
 \hline
 T & F & T & F & F & F & F & T & F\\
 \hline
 T & F & F & F & F & F & F & T & F\\
 \hline
 F & T & T & T & T & T & T & F & F\\
 \hline
 F & T & F & T & T & F & F & F & F\\
 \hline
 F & F & T & T & F & F & F & T & F\\
 \hline
 F & F & F & T & F & F & F & T & F\\
 \hline
\end{tabular}
\end{center}

\end{addmargin}
\bigskip

18.
\begin{addmargin}[1em]{2em}
contradication이다.

\end{addmargin}
\bigskip

19.
\begin{addmargin}[1em]{2em}
contradication은 statement variable로 치환된 개별적인 statement의 진리값에 상관없이, 언제나 거짓인 statement form이다. 17번의 truth table로 미루어보았을 때, statement variable p와 q, 그리고 r의 진리값으로 가능한 모든 경우에 있어 이 statement form의 진리값은 거짓이므로, contradication이다.

\end{addmargin}
\bigskip

20.
\begin{addmargin}[1em]{2em}
\begin{center}
\begin{tabular}{ |c|c|c|c|c|c|c| } 
 \hline
 p & q & r & p \wedge q \to \sim r & p \vee \sim q & \sim q \to p & \sim r \\ 
  \hline
 T & T & T & F & T & T & F\\
  \hline
 T & T & F & T & T & T & T\\
 \hline
 T & F & T & T & T & T & F\\
 \hline
 T & F & F & T & T & T & T\\
 \hline
 F & T & T & T & F & T & F\\
 \hline
 F & T & F & T & F & T & T\\
 \hline
 F & F & T & T & T & F & F\\
 \hline
 F & F & F & T & T & F & T\\
 \hline
\end{tabular}
\end{center}
네번째부터 여섯번째 column은 전제, 일곱번째 column은 결론을 나타낸다.

\end{addmargin}
\bigskip

21.
\begin{addmargin}[1em]{2em}
valid하지 않다.

\end{addmargin}
\bigskip

22.
\begin{addmargin}[1em]{2em}
argument form이 valid하다는 것은 전제가 모두 참이라면 결론 역시 참이라는 것을 뜻한다. 그런데 20번의 truth table의 4번째 row는 전제가 모두 참이지만 결론이 거짓인 경우를 보여주고 있다. 따라서 이 argument form은 valid하지 않다.

\end{addmargin}
\bigskip

23.
\begin{addmargin}[1em]{2em}
\(V\) : all who can tell a valid argument from an invalid one.

\(D\) : all discrete mathematics students

\(T\) : all thoughtful people.

\def\firstcircle{(0,0) circle (3cm)}
\def\secondcircle{(0:0.5cm) circle (1.5cm)}

\begin{tikzpicture}
    \draw \firstcircle;
    \draw \secondcircle;
    \node at (0.4,0) {\LARGE\(D\)};
    \node at (-1.0,1.5) {\LARGE\(V\)};
    \node at (0.0, -3.5){(a) Major premise};
    \begin{scope}
      \clip \firstcircle;
      \clip \secondcircle;
    \end{scope}
\end{tikzpicture}
\bigskip

\begin{tikzpicture}
    \draw \firstcircle;
    \draw \secondcircle;
    \node at (0.4,0) {\LARGE\(T\)};
    \node at (-1.0,1.5) {\LARGE\(V\)};
    \node at (0.0, -3.5){(b) Minor premise};
    \begin{scope}
      \clip \firstcircle;
      \clip \secondcircle;
    \end{scope}
\end{tikzpicture}

\begin{tikzpicture}
    \draw \firstcircle;
    \draw \secondcircle;
    \node at (0.4,0) {\LARGE\(D\)};
    \node at (-1.0,1.5) {\LARGE\(T\)};
    \node at (0.0, -3.5){(c) Conclusion};
    \begin{scope}
      \clip \firstcircle;
      \clip \secondcircle;
    \end{scope}
\end{tikzpicture}

\end{addmargin}
\bigskip

24.
\begin{addmargin}[1em]{2em}
valid하지 않다.

\end{addmargin}
\bigskip

25.
\begin{addmargin}[1em]{2em}
argument form이 valid하다는 것은 전제가 모두 참이라면 결론 역시 참이라는 것을 뜻한다. 그런데 23번의 다이어그램을 통해 \(x \in V\)이면서 \(x \in D\) 이고,  \(x \notin T\) 인 어떤 \(x\) 가 존재할 수 있다는 것을 알 수 있다. 따라서 전제가 모두 참이면서 결론이 거짓일 수 있으므로, valid 하지 않다.

\end{addmargin}
\bigskip

26.
\begin{addmargin}[1em]{2em}
negation of universal statement의 정의에 따르면, (1)와 같은 형태의 universal statement의 negation은 (2)와 같은 형태의 statement와 logically equivalent하다.

\begin{align*}
&\forall x\mathrm{\ in\ }D,\ Q(x) &(1) \\
&\exists x\mathrm{\ in\ }D \mathrm{\ such\ that\ }\sim Q(x) &(2) \\
\end{align*}

그러므로 주어진 form의 좌변은 다음과 같이 나타낼 수 있다.
\begin{align*}
&\sim(\forall x,\mathrm{if\ }P(x)\mathrm{\ then\ }Q(x)) \equiv \exists x\mathrm{\ such\ that\ }\sim (\mathrm{if\ }P(x)\mathrm{\ then\ }Q(x))) &(a) \\
\end{align*}

한편, if-staement의 negation은 logically equivalent한 and statement로 표현할 수 있다.
\begin{align*}
&\sim(\mathrm{if\ }P(x)\mathrm{\ then\ }Q(x)) \equiv P(x)\mathrm{\ and\ }\sim Q(x) &(b) \\
\end{align*}

(b)의 우변으로 (a)의 우변를 바꾸면 다음과 같다.
\begin{align*}
&\sim(\forall x,\mathrm{if\ }P(x)\mathrm{\ then\ }Q(x)) \equiv \exists x\mathrm{\ such\ that\ }P(x)\mathrm{\ and\ }\sim Q(x) &(c) \\
\end{align*}
\end{addmargin}
\bigskip
\end{document}

